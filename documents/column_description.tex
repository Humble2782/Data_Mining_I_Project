% Do not change document class, margins, fonts, etc.
\documentclass[a4paper,oneside,bibliography=totoc]{scrbook}

% some useful packages (add more as needed)
\usepackage[utf8]{inputenc}
\usepackage{graphicx}
\usepackage{latexsym}
\usepackage{amsmath}
\usepackage{amssymb}
\usepackage{tabularx}
\usepackage{booktabs}
\usepackage{algorithm} % you can modify the algorithm style to your liking
\counterwithin{algorithm}{chapter} % so that algorithms have chapter numbers as well
\usepackage{algorithmic}
\usepackage{csquotes}
\renewcommand{\algorithmiccomment}[1]{\hfill\textit{// #1}}
\usepackage[usenames,dvipsnames]{xcolor}
\usepackage[colorlinks,citecolor=Green]{hyperref} % you may change/remove the colors
\usepackage{lipsum} % you do not need this
\usepackage[T1]{fontenc} % For better font encoding and handling of underscores
\usepackage{lmodern} % Modern LaTeX font

% chicago citation style
\usepackage{natbib}
\bibliographystyle{chicagoa}
\setcitestyle{authoryear,round,semicolon,aysep={},yysep={,}} \let\cite\citep

% example enviroments (add more as needed)
\newtheorem{definition}{Definition} \newtheorem{proposition}{Proposition}

% Use \texttt for code/column names to handle underscores correctly
% --- FIX ---
% Changed \texttt{#1} to \texttt{\detokenize{#1}}
% This prevents LaTeX from treating underscores (_) as math characters
% and fixes the compilation error.
\newcommand{\col}[1]{\texttt{\detokenize{#1}}}

\begin{document}


% Quick fix to remove the 0 in front of sections.
\makeatletter
\renewcommand\thesection{\arabic{section}}
\renewcommand\thesubsection{\arabic{section}.\arabic{subsection}}
\renewcommand\thesubsubsection{\arabic{section}.\arabic{subsection}.\arabic{subsubsection}}
\makeatother



\frontmatter \subject{Predicting Injury Severity in Road Accidents: A Real-Time Classification Approach} % change to appropriate type
\title{Feature Description}
\author{
  David Cebulla (1922129)\\
  Gabriel Himmelein (1649181)\\
  Lukas Ott (1842341)\\
  Artur Loreit (2268917)\\
  Aaron Niemesch (1836924)
} \date{October 25, 2025}
\publishers{{\small Submitted to}\\
  Data and Web Science Group\\
  Dr.\ Sven Hertling\\
  University of Mannheim\\}
\maketitle

\mainmatter
\section{Introduction}
This document describes all features available in the final dataset (after the \texttt{F\_feature\_selection} step) used for model training. The features are divided into two main groups:

\begin{enumerate}
    \item \textbf{Original Features:} Columns originating from the raw datasets (renamed in \texttt{a\_rename.py}) that were retained.
    \item \textbf{Engineered Features:} New columns calculated in \texttt{c\_feature\_engineering.py} or \texttt{b\_merge\_tables.py} to model complex relationships.
\end{enumerate}

Both groups are further subdivided by data type: \textbf{Nominal} (Categorical, no order), \textbf{Ordinal} (Categorical, with order), and \textbf{Numerical} (Continuous).

\section{Target Variable (The Feature to Predict)}
This is the single, engineered column that the model is trained to predict.

\subsection{Ordinal}
\begin{itemize}
    \item \textbf{\col{injury_target}}: (Engineered) Our new, ordinal target variable representing the severity of the injury. It is derived from the original \col{injury_severity} column.
        \begin{itemize}
            \item \texttt{0}: Uninjured (from Original: 1)
            \item \texttt{1}: Lightly Injured (from Original: 4)
            \item \texttt{2}: Severe (Hospitalized or Killed) (from Original: 2, 3)
        \end{itemize}
\end{itemize}

\section{Original Features (Retained from Source Data)}
These columns are loaded from the raw data, renamed, and are retained after all processing and feature selection steps.

\subsection{Nominal Features (Categorical)}
\begin{itemize}
    \item \textbf{\col{location}}: Location of the accident. \textit{(Original: \col{agg})}
        \begin{itemize}
            \item \texttt{1}: Outside built-up area
            \item \texttt{2}: In built-up area
        \end{itemize}
    \item \textbf{\col{type_of_collision}}: Type of collision. \textit{(Original: \col{col})}
        \begin{itemize}
            \item \texttt{-1}: Not specified (imputed)
            \item \texttt{1}: Two vehicles - frontal
            \item \texttt{2}: Two vehicles - from behind
            \item \texttt{3}: Two vehicles - from the side
            \item \texttt{4}: Three vehicles and more - in chain
            \item \texttt{5}: Three vehicles and more - multiple collisions
            \item \texttt{6}: Other collision
            \item \texttt{7}: Without collision
        \end{itemize}
    \item \textbf{\col{reserved_lane_present}}: Indicates the existence of a reserved lane. \textit{(Original: \col{vosp})}
        \begin{itemize}
            \item \texttt{-1}: Not specified
            \item \texttt{0}: Not applicable
            \item \texttt{1}: Bicycle path
            \item \texttt{2}: Bicycle lane
            \item \texttt{3}: Reserved lane
        \end{itemize}
    \item \textbf{\col{horizontal_alignment}}: Horizontal alignment (plan) of the road. \textit{(Original: \col{plan})}
        \begin{itemize}
            \item \texttt{-1}: Not specified
            \item \texttt{1}: Straight section
            \item \texttt{2}: Left curve
            \item \texttt{3}: Right curve
            \item \texttt{4}: In "S" shape
        \end{itemize}
    \item \textbf{\col{infrastructure}}: Special infrastructure at the accident site. \textit{(Original: \col{infra})}
        \begin{itemize}
            \item \texttt{-1}: Not specified
            \item \texttt{0}: None
            \item \texttt{1}: Underground - tunnel
            \item \texttt{2}: Bridge - overpass
            \item \texttt{3}: Interchange ramp or connection
            \item \texttt{4}: Railway
            \item \texttt{5}: Equipped junction
            \item \texttt{6}: Pedestrian zone
            \item \texttt{7}: Toll zone
            \item \texttt{8}: Construction site
            \item \texttt{9}: Others
        \end{itemize}
    \item \textbf{\col{accident_situation}}: Situation of the accident. \textit{(Original: \col{situ})}
        \begin{itemize}
            \item \texttt{-1}: Not specified
            \item \texttt{0}: None
            \item \texttt{1}: On roadway
            \item \texttt{2}: On emergency lane
            \item \texttt{3}: On shoulder
            \item \texttt{4}: On sidewalk
            \item \texttt{5}: On bicycle path
            \item \texttt{6}: On other special lane
            \item \texttt{8}: Others
        \end{itemize}
    \item \textbf{\col{sex}}: Sex of the user. \textit{(Original: \col{sexe})}
        \begin{itemize}
            \item \texttt{1}: Male
            \item \texttt{2}: Female
        \end{itemize}
    \item \textbf{\col{pedestrian_location}}: Location of the pedestrian at the time of the accident. \textit{(Original: \col{locp})}
        \begin{itemize}
            \item \texttt{-1}: Not specified
            \item \texttt{0}: Not applicable
            \item \texttt{1}: On roadway, >50m from pedestrian crossing
            \item \texttt{2}: On roadway, <50m from pedestrian crossing
            \item \texttt{3}: On pedestrian crossing, without traffic light
            \item \texttt{44}: On pedestrian crossing, with traffic light
            \item \texttt{5}: On sidewalk
            \item \texttt{6}: On shoulder
            \item \texttt{7}: On refuge or emergency lane
            \item \texttt{8}: On parallel lane
            \item \texttt{9}: Unknown
        \end{itemize}
    \item \textbf{\col{pedestrian_action}}: Action of the pedestrian. \textit{(Original: \col{actp})}
        \begin{itemize}
            \item \texttt{-1}: Not specified
            \item \texttt{0}: Not specified or not applicable
            \item \texttt{1}: Moving in the same direction as the striking vehicle
            \item \texttt{2}: Moving in the opposite direction of the vehicle
            \item \texttt{3}: Crossing
            \item \texttt{4}: Masked / Hidden
            \item \texttt{5}: Playing - running
            \item \texttt{6}: With animal
            \item \texttt{9}: Other
            \item \texttt{A}: Getting on/off vehicle
            \item \texttt{B}: Unknown
        \end{itemize}
    \item \textbf{\col{fixed_obstacle_struck}}: Fixed obstacle struck by the primary vehicle. \textit{(Original: \col{obs})}. Imputed with -1 (N/A) for pedestrians.
        \begin{itemize}
            \item \texttt{-1}: Not specified / N/A
            \item \texttt{0}: Not applicable
            \item \texttt{1}: Parked vehicle
            \item \texttt{2}: Tree
            \item \texttt{3}: Metal guard rail
            \item \texttt{4}: Concrete guard rail
            \item \texttt{5}: Other guard rail
            \item \texttt{6}: Building, wall, bridge pier
            \item \texttt{7}: Road sign support or emergency call box
            \item \texttt{8}: Post
            \item \texttt{9}: Street furniture
            \item \texttt{10}: Parapet
            \item \texttt{11}: Island, refuge, high bollard
            \item \texttt{12}: Curb
            \item \texttt{13}: Ditch, embankment, rock wall
            \item \texttt{14}: Other fixed obstacle on roadway
            \item \texttt{15}: Other fixed obstacle on sidewalk or shoulder
            \item \texttt{16}: Road exit without obstacle
            \item \texttt{17}: Culvert
        \end{itemize}
    \item \textbf{\col{mobile_obstacle_struck}}: Mobile obstacle struck by the primary vehicle. \textit{(Original: \col{obsm})}. Imputed with -1 (N/A) for pedestrians.
        \begin{itemize}
            \item \texttt{-1}: Not specified / N/A
            \item \texttt{0}: None
            \item \texttt{1}: Pedestrian
            \item \texttt{2}: Vehicle
            \item \texttt{4}: Vehicle on rail
            \item \texttt{5}: Domestic animal
            \item \texttt{6}: Wild animal
            \item \texttt{9}: Other
        \end{itemize}
    \item \textbf{\col{initial_point_of_impact}}: Initial point of impact on the primary vehicle. \textit{(Original: \col{choc})}. Imputed with -1 (N/A) for pedestrians.
        \begin{itemize}
            \item \texttt{-1}: Not specified / N/A
            \item \texttt{0}: None
            \item \texttt{1}: Front
            \item \texttt{2}: Front right
            \item \texttt{3}: Front left
            \item \texttt{4}: Rear
            \item \texttt{5}: Rear right
            \item \texttt{6}: Rear left
            \item \texttt{7}: Right side
            \item \texttt{8}: Left side
            \item \texttt{9}: Multiple impacts (rollover)
        \end{itemize}
    \item \textbf{\col{main_maneuver_before_accident}}: Main maneuver of the primary vehicle. \textit{(Original: \col{manv})}. Imputed with -1 (N/A) for pedestrians/passengers, 0 (Unknown) for drivers.
        \begin{itemize}
            \item \texttt{-1}: Not specified / N/A
            \item \texttt{0}: Unknown
            \item \texttt{1}: Without change of direction
            \item \texttt{...} (Full list in original file)
            \item \texttt{26}: Other maneuvers
        \end{itemize}
    \item \textbf{\col{motor_type}}: Motorization type of the primary vehicle. \textit{(Original: \col{motor})}. Imputed with -1 (N/A) for pedestrians/passengers, 0 (Unknown) for drivers.
        \begin{itemize}
            \item \texttt{-1}: Not specified / N/A
            \item \texttt{0}: Unknown
            \item \texttt{1}: Hydrocarbon (Gasoline/Diesel)
            \item \texttt{2}: Hybrid electric
            \item \texttt{3}: Electric
            \item \texttt{4}: Hydrogen
            \item \texttt{5}: Human (e.g., bicycle)
            \item \texttt{6}: Other
        \end{itemize}
    
    \item \textbf{\col{direction_of_travel_other}}, \textbf{\col{fixed_obstacle_struck_other}}, \textbf{\col{mobile_obstacle_struck_other}}, \textbf{\col{initial_point_of_impact_other}}, \textbf{\col{main_maneuver_before_accident_other}}, \textbf{\col{motor_type_other}}:
    \textit{Note on \col{_other} columns:} These features describe the \textbf{highest-impact "other" vehicle} involved in the accident (determined in \texttt{b\_merge\_tables.py}). They follow the same code definitions as their primary counterparts. In \texttt{d\_handle\_missing\_values.py}, \texttt{NaN} values are imputed with \textbf{\texttt{-1}} (N/A) if no second vehicle was involved, or \textbf{\texttt{0}} (Unknown) if a second vehicle was present but the data was missing.

\end{itemize}

\subsection{Ordinal Features (Ordered Categories)}
\begin{itemize}
    \item \textbf{\col{position}}: Position occupied by the user in the vehicle. \textit{(Original: \col{place})}. (e.g., \texttt{1}: Driver, \texttt{2}-\texttt{9}: Passenger seats).
\end{itemize}

\subsection{Numerical Features (Continuous)}
\begin{itemize}
    \item \textbf{\col{latitude}}: Latitude (WGS84). \textit{(Original: \col{lat})}
    \item \textbf{\col{longitude}}: Longitude (WGS84). \textit{(Original: \col{long})}
    \item \textbf{\col{speed_limit}}: Authorized speed limit at the accident site. \textit{(Original: \col{vma})}
\end{itemize}


\section{Engineered Features (Created by the Pipeline)}
These columns are newly calculated in \texttt{c\_feature\_engineering.py} or \texttt{b\_merge\_tables.py}.

\subsection{Nominal Features (Categorical)}
\begin{itemize}
    \item \textbf{\col{time_of_day}}: Categorical time bucket derived from \col{hour}.
        \begin{itemize}
            \item \texttt{'Night'}: 00:00 - 05:59, 20:00 - 23:59
            \item \texttt{'Morning\_Rush'}: 06:00 - 09:59
            \item \texttt{'Midday'}: 10:00 - 15:59
            \item \texttt{'Evening\_Rush'}: 16:00 - 19:59
        \end{itemize}
    \item \textbf{\col{age_group}}: Age bracket derived from \col{age}.
        \begin{itemize}
            \item \texttt{'child\_teen'}: 0-17
            \item \texttt{'young\_adult'}: 18-24
            \item \texttt{'adult'}: 25-39
            \item \texttt{'middle\_aged'}: 40-64
            \item \texttt{'senior'}: 65+
            \item \texttt{'Unknown'}: Imputed value
        \end{itemize}
    \item \textbf{\col{role}}: Simplified user role, derived from \col{user_category}.
        \begin{itemize}
            \item \texttt{'driver'}: (Original: 1)
            \item \texttt{'passenger'}: (Original: 2)
            \item \texttt{'pedestrian'}: (Original: 3)
            \item \texttt{'other'}: (Imputed value)
        \end{itemize}
    \item \textbf{\col{vehicle_category_simplified}}: Simplified vehicle category for the primary vehicle. (e.g., \texttt{'light\_motor\_vehicle'}, \texttt{'hgv\_truck'}, \texttt{'bicycle'}, \texttt{'unknown'}, etc.)
    \item \textbf{\col{vehicle_category_simplified_other}}: Simplified vehicle category for the \textit{other} vehicle. (Includes \texttt{'n/a'} for no other vehicle).
    \item \textbf{\col{used_belt}}: (Binary) 1 if user used a seatbelt, 0 otherwise.
    \item \textbf{\col{used_helmet}}: (Binary) 1 if user used a helmet, 0 otherwise.
    \item \textbf{\col{used_child_restraint}}: (Binary) 1 if a child restraint was used, 0 otherwise.
    \item \textbf{\col{used_airbag}}: (Binary) 1 if airbag was deployed/used, 0 otherwise.
    \item \textbf{\col{surface_quality_indicator}}: (Binary) 1 if \col{pavement_condition} = "Normal" (1) AND \col{longitudinal_profile} = "Flat" (1), 0 otherwise.
    
    \item \textbf{\col{vehicle_category_involved_bicycle}}: (Binary) 1 if at least one bicycle was involved in the accident, 0 otherwise.
    \item \textbf{\col{vehicle_category_involved_bus_coach}}: (Binary) 1 if a bus/coach was involved.
    \item \textbf{\col{vehicle_category_involved_hgv_truck}}: (Binary) 1 if a heavy truck was involved.
    \item \textbf{\col{vehicle_category_involved_light_motor_vehicle}}: (Binary) 1 if a light motor vehicle/car was involved.
    \item \textbf{\col{vehicle_category_involved_other}}: (Binary) 1 if an "other" vehicle type was involved.
    \item \textbf{\col{vehicle_category_involved_powered_2_3_wheeler}}: (Binary) 1 if a moped/motorcycle was involved.
\end{itemize}

\subsection{Ordinal Features (Ordered Categories)}
\begin{itemize}
    \item \textbf{\col{day_of_week}}: Day of the week, where Monday=0 and Sunday=6.
    \item \textbf{\col{lighting_ordinal}}: A new ordinal scale for lighting conditions (risk-based).
        \begin{itemize}
            \item \texttt{0}: Good (Original: 1 - Full day)
            \item \texttt{1}: Medium (Original: 5 - Night, light on)
            \item \texttt{2}: Poor (Original: 2 - Twilight)
            \item \texttt{3}: Very Poor (Original: 3, 4 - Night, light off/none)
        \end{itemize}
    \item \textbf{\col{weather_ordinal}}: A new ordinal scale for weather conditions (risk-based).
        \begin{itemize}
            \item \texttt{0}: Good (Original: 1 - Normal)
            \item \texttt{1}: Okay (Original: 8 - Overcast)
            \item \texttt{2}: Slight Risk (Original: 2 - Light rain, 7 - Dazzling)
            \item \texttt{3}: Medium Risk (Original: 6 - Wind, 3 - Heavy rain)
            \item \texttt{4}: High Risk (Original: 5 - Fog, 4 - Snow)
        \end{itemize}
    \item \textbf{\col{road_complexity_index}}: Index (scaled 0-10) assessing road complexity. Based on \col{intersection}, \col{road_category}, \col{traffic_regime}, and \col{number_of_traffic_lanes}. Higher value = more complex.
    \item \textbf{\col{impact_score}}: Weighted score (0-6) based on \col{vehicle_category_simplified}. (e.g., Truck=6, Car=4, Bicycle=2, unknown=1).
    \item \textbf{\col{impact_score_other}}: Weighted score (0-6) for the \textit{other} vehicle. (e.g., Truck=6, Car=4, n/a=0).
    \item \textbf{\col{impact_delta}}: The \textit{directional} difference: \col{impact_score} - \col{impact_score_other}. A negative value implies higher risk (e.g., Car vs. Truck = 4 - 6 = -2).
\end{itemize}

\subsection{Numerical Features (Continuous)}
\begin{itemize}
    \item \textbf{\col{age}}: User's age at the time of the accident. Imputed with 0 if unknown.
    \item \textbf{\col{hour_sin}} / \textbf{\col{hour_cos}}: Cyclical features (Sine/Cosine) for the hour of the day.
    \item \textbf{\col{day_of_week_sin}} / \textbf{\col{day_of_week_cos}}: Cyclical features for the day of the week.
    \item \textbf{\col{month_sin}} / \textbf{\col{month_cos}}: Cyclical features for the month.
    \item \textbf{\col{day_of_year_sin}} / \textbf{\col{day_of_year_cos}}: Cyclical features for the day of the year.
\end{itemize}

\end{document}

